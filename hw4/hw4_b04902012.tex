\documentclass{report}
\usepackage{hyperref}
\usepackage{fontspec,xunicode}
\usepackage{amsmath} 
\usepackage{graphicx}
\usepackage{nopageno}
\usepackage{ulem}
\setmainfont{AR PL UKai TW}
\usepackage{CJK}
\XeTeXlinebreaklocale "zh"
\usepackage{type1cm}
\begin{document}
\pagestyle{plain}
\fontsize{12pt}{20pt}\selectfont
\begin{center}
	\Huge{計算機結構 作業四}\\ 
	\huge{B04902012 劉瀚聲}\\ 
	\huge{2016.11.20}\\ 
\end{center}
	
\noindent{\huge Coding Environment}\\
  這個專案在 \texttt{Linux 4.13.11-1} 上被編程和測試運行。在本地經 \texttt{iverilog} 編譯、\texttt{vpp} 執行,與 \texttt{output.txt} 核對後正確無誤。\\

\noindent{\huge Module Implementation}\\
  這個專案以 \texttt{CPU.v} 為核心,遵循 \texttt{hw4.pdf} 中所給定的 datapath,在 module \texttt{CPU} 中以 wire 將各個元件連接在一起。\\
  ALU Operation Code 和 ALU Control Code 皆與實際使用的值一致。唯元件 \texttt{ALU\_Control} 中,由於官方文件並沒有給出 \texttt{mult} 的 ALU Control Code,我擅自使用 \texttt{011} 代替之。\\

\end{document} 
